\documentclass[12pt,a4paper]{article}
\usepackage{alltt}
\usepackage{amsfonts}
\usepackage{amsmath}
\usepackage{amsthm}
\usepackage{attrib}
\usepackage{cite}
\usepackage{color}
\usepackage{fancyhdr}
\usepackage{fancyvrb}
\usepackage{float}
\usepackage{fullpage}
\usepackage{graphicx}
\usepackage{setspace}
\usepackage{xspace}
\usepackage{style/code}
\usepackage{style/utils}
\usepackage{url}
% \usepackage{times}
\newcommand{\at}{\texttt{@}}
\makeatactive

\renewcommand{\labelenumii}{\theenumii}
\renewcommand{\theenumii}{\theenumi.\arabic{enumii}.}

\floatstyle{plain}
\newfloat{codefig}{H}

\newcommand{\codefigend}{\vspace{-45pt}}


% \usepackage{graphicx}
\usepackage{fancyvrb}
\usepackage{amsmath}
\usepackage{amsthm}
\usepackage{fullpage}
\usepackage{float}
\usepackage{alltt}
\usepackage{color}

\floatstyle{plain}
\newfloat{codefig}{H}

\newcommand{\codefigend}{\vspace{-45pt}}

\numberwithin{section}{chapter}
\theoremstyle{plain}    
\newtheorem{thm}{Theorem}[chapter]
\numberwithin{figure}{chapter} %% Comment out for sequentially-numbered
\numberwithin{table}{chapter}  %% Comment out for sequentially-numbered

%-----------------------------------------------------------------------
%  Reduce widow/orphan problems, mainly from a posting from Donald
%  Arsenau on comp.text.tex, 24 Sep 1995.
%  Updated to follow comments from Michael Downes on comp.text.tex,
%  31 Aug 1998.
%-----------------------------------------------------------------------
\doublehyphendemerits=10000     % No consecutive line hyphens.
\brokenpenalty=4991             % Reduce broken words across columns/pages.
\widowpenalty=9999              % Almost no widows at bottom of page.
\clubpenalty=9996               % Almost no orphans at top of page.
\interfootnotelinepenalty=9999  % Almost never break footnotes.
\predisplaypenalty=10000        % Default value
\postdisplaypenalty=1549        % Few breaks between display and widows
\displaywidowpenalty=1602       % At least as high as \postdisplaypenalty

%-----------------------------------------------------------------------
% Change float placement parameters to reduce problems.  Based on
% values posted by Donald Arsenau on comp.text.tex at various times.
% See in particular 17th Nov 1997.
%-----------------------------------------------------------------------
\renewcommand{\topfraction}{.85}
\renewcommand{\bottomfraction}{.7}
\renewcommand{\textfraction}{.15}
\renewcommand{\floatpagefraction}{.66}
\renewcommand{\dbltopfraction}{.66}
\renewcommand{\dblfloatpagefraction}{.66}
\setcounter{topnumber}{9}
\setcounter{bottomnumber}{9}
\setcounter{totalnumber}{20}
\setcounter{dbltopnumber}{9}

%-----------------------------------------------------------------------
% Re-define \cleardoublepage as recommended by Piet van Oostrum in the
% documentation for fancyhdr.sty page 15.  This is to avoid blank pages
% having headers or footers.
%-----------------------------------------------------------------------
\renewcommand{\cleardoublepage}{\clearpage\if@twoside \ifodd\c@page\else
   \thispagestyle{empty}
   \hbox{}\newpage\if@twocolumn\hbox{}\newpage\fi\fi\fi}

%-----------------------------------------------------------------------
% Number figures, tables and equations by chapter.  Re-define footnotes
% and minipage footnotes to be single spaced.  Make new macros needed
% for thesis definitions.
%-----------------------------------------------------------------------
\renewcommand{\thefigure}{\thechapter.\arabic\c@figure}
\renewcommand{\thetable}{\thechapter.\arabic\c@table}
\renewcommand{\theequation}{\thechapter.\arabic\c@equation}

% Re-define \@footnotetext and \@mpfootnotetext to use single spacing
% rather than the space-and-a-half that is the default elsewhere.

\renewcommand{\bibname}{References}

\renewcommand{\title}[1]{\gdef\title{#1}}
\renewcommand{\author}[1]{\gdef\author{#1}}
\newcommand{\school}[1]{\gdef\school{#1}}
\newcommand{\degreetype}[1]{\gdef\degreetype{#1}}
\newcommand{\submitdate}[1]{\gdef\submitdate{#1}}

\renewcommand{\title}{}
\renewcommand{\author}{}
\renewcommand{\school}{Computer Science and Engineering}
\renewcommand{\degreetype}{Bachelor of Computer Science (Hons)}
\newcommand{\university}{The University of New South Wales}
\renewcommand{\submitdate}{\ifcase\the\month\or
  January\or February\or March\or April\or May\or June\or
  July\or August\or September\or October\or November\or December\fi
  \space \number\the\year}
\newcommand{\copyrightyear}{\number\the\year}

%-----------------------------------------------------------------------
% Title page definition.
%-----------------------------------------------------------------------
\newcommand{\titlep}{%
        \thispagestyle{empty}%
        \begin{titlepage}
        \null\vskip2.5cm%
        \begin{center}
                {\rmfamily\Large\expandafter{\title}}
        \end{center}
        \vskip1cm
        \begin{center}
                {\rmfamily\normalsize By \\ \expandafter{\author}}
        \end{center}
        \vfill
        \begin{center}
                \textsc{A thesis submitted for the degree of \\
                \expandafter{\degreetype}}
        \end{center}
        \vfill
        \vskip1cm
\begin{center}
\includegraphics[width=4cm]{template/unswcrest.pdf}
\end{center}
        \vskip1cm
        \begin{center}
                {\rmfamily\normalsize \expandafter{\school},\\
                \expandafter{\university}.}
                \vskip1cm
                {\rmfamily\normalsize \submitdate}\\
        \end{center}
        \vskip1cm
        \end{titlepage}
        \newpage}

%-----------------------------------------------------------------------
% Redefine \maketitle so as to make a UNSW suitable title
%-----------------------------------------------------------------------
\renewcommand{\maketitle}{%
        \thispagestyle{empty}%
        \newpage%
        \titlep
        \raggedbottom
    }




\pagestyle{fancy}
%\markboth{Amos Robinson}{3400438}
\lhead{}
\chead{}
\rhead{}
\lfoot{Amos Robinson, 3400438}
\cfoot{}
\rfoot{\thepage}
\renewcommand{\headrulewidth}{0pt}

\author{Amos Robinson, 3400438}
\title{Fixing flattening's space complexity with flow fusion}

\begin{document}

\maketitle
\thispagestyle{fancy}

%\maketitle
\begin{abstract}
Writing parallel programs is fraught with danger; deadlocks, heisenbugs and race conditions lurk at every corner case.
To alleviate the programmer from this pain, techniques such as flat data parallelism and nested data parallelism can be used.
Flat data parallelism can be implemented quite efficiently, but only supports regular shaped arrays such as rectangles.
Nested data parallelism removes this restriction, but requires quite a bit more work to implement efficiently.

The flattening transform is core to translating nested data parallelism to flat parallelism.
However, in some cases flattening can ruin the space and work complexity of a program.
The work complexity problem has since been solved by using a clever representation of arrays, but the space complexity problem remains.

We believe that our new development, flow fusion, will be able to solve some instances of the space complexity problem.
In addition, we aim to investigate and research the extra fusion required to solve the space complexity problem completely.
If the problem cannot be solved completely, the cases that cannot be solved will be formalised and compiler warnings issued in those cases.
\end{abstract}
\pagebreak

% actual content must be 1.5line-spacing. make ToC 1.5 too, because normal looks silly if spanning *just* two pages
\onehalfspacing
\tableofcontents

\newcommand{\nesl}{{\sc Nesl}\xspace}
\newcommand{\vcode}{{\sc Vcode}\xspace}

% \section{Submission cover sheet}
\includegraphics{submissioncoversheet.png}


\section{Introduction}
As transistors continue to decrease in size, as they have been for the last fifty years, more transistors are able to fit into a single processor.
Historically this has led to faster processors, allowing programs to take advantage of the increased speed with no effort from the programmer.
Recently, however, this increase in processor speed has come to a halt.
Being unable to increase processor speed, manufacturers have instead started focussing the new transistors on multi-core processors.

The ubiquity of multi-core processors means that sequential programs can no longer take full advantage of the processor.
However, writing parallel programs is significantly more difficult than writing sequential programs.
Nested data parallelism lifts the burden of exploiting multi-cores from the programmer to the compiler-writer.
Current implementations of nested data parallelism use impressive optimisation techniques such as fusion, but are still unable to compete with handwritten parallel code.
The aim of my research is to improve the performance of nested data parallelism, making it easier for programmers to write efficient parallel code.
Because nested data parallelism has focussed on purely functional languages, many well-known imperative optimisations have not been treated.

Purely functional languages, such as Haskell, are more restricted than impure imperative languages
in the sense that arbitrary expressions may not perform side-effects such as destructive updates.
This restriction, however, turns out to have serious benefits:
an optimising compiler is able to reorder expressions to gain better performance,
without any risk of changing the meaning of the program.
From an engineering perspective also, the separation of side-effects from pure computation leads to clearer, easier to understand and verify code.

This document aims to review the current state of automatic parallelism techniques, nested data parallelism in particular,
and the optimisations required to get them running at acceptable speeds.
My overall dream is to have parallel programs that are easier to write, and execute faster than their C equivalents.
So far only the first of those dreams has been realised, with Data Parallel Haskell (DPH)
being able to express data parallel programs clearly and succinctly.
These programs, while running significantly faster than list-based Haskell programs, cannot generally compete with hand-optimised C code.


\pagebreak
\section{Literature review}
\section{Parallelism}
Writing parallel programs that operate on a large amount of data can be difficult.
To achieve the best performance on a multi-processor machine, programs must be written using low-level synchronisation techniques such as threads and locking.
These methods are error prone, and improper use can lead to deadlocks and Heisenbugs: such errors are inherently hard to debug.

There are several techniques for making parallelism easier to take advantage of,
each with their own advantages and disadvantages.
We discuss three techniques:

\begin{description}
\item[Flat   data parallelism]
is able to achieve speeds comparable to C with advanced fusion techniques,
but only supports regular structures, such as rectangular and cubic arrays.

\item[Nested data parallelism]
is similar to flat data parallelism,
but supports parallel operations over tree-like structures
while keeping the load balanced over processors.

\item[Feedback directed automatic parallelism]
is a different beast: it uses profiling information from running the program
to find the best places to introduce parallelism. 
\end{description}


\subsection{Flat data parallelism}
Flat data parallelism is a restricted form of data parallelism that disallows nested parallelism:
only the top-most computation is performed in parallel.
To ensure a balanced load across parallel computations, only regular structures such as rectangular arrays can be used.
Programs are expressed using combinators such as @map@, @filter@, @sum@ and @fold@.

Accelerate is an \emph{embedded domain-specific language} in Haskell, specifically for running parallel computations on graphics processors.
As a restricted domain-specific language, it targets graphics processors\cite{chakravarty2011accelerating} and uses fusion to achieve speeds similar to CUDA\cite{mcdonell2013optimising}.

Repa is a Haskell library with combinator support for regular parallel array computations\cite{keller2010regular}.
Type indices are used to distinguish between so-called \emph{delayed arrays} and \emph{manifest arrays}.
Delayed arrays will not be reified in memory at runtime, but are instead expressed as functions of the array index.
With sufficient inlining and other general purpose optimisations, delayed arrays will be removed and fused together\cite{lippmeier2012guiding}.
Efficient stencil convolution is supported as a special case.
To remove branching from the worker loops, the edges of a stencil are computed in different loops\cite{lippmeier2011efficient}.

Many algorithms are more naturally expressed using nested parallelism, such as the hierarchical N-body calculation\cite{barnes1986hierarchical} algorithm and sparse-matrix vector multiplication.
For these algorithms, flat data parallelism is insufficient.


\subsection{Nested data parallelism}
Nested data parallelism, as introduced in a seminal paper by Blelloch\cite{blelloch1990vector},
uses a transform known as \emph{flattening} to convert nested parallelism into flat data parallelism
while maintaining a balanced load across processors.

A language specifically for nested data parallelism, \nesl\cite{blelloch1995nesl},
uses the flattening transform to remove nested parallelism.
It targets a virtual machine called \vcode\cite{blelloch1993implementation}
which is then compiled to C to run on vector machines such as CRAY.
Many example programs have been implemented in \nesl\cite{blelloch1996programming}
but it is not a general purpose language and does not support higher-order functions\cite{leshchinskiy2005higher} or recursive structures \cite{keller1998flattening}.

Attempts have been made to introduce nested data parallelism to imperative languages such as Fortran\cite{au1997enlarging}.
These have the benefit of taking advantage of existing compiler optimisations.
The problem with targeting an impure language, though,
is that arbitrary expressions may perform side-effects.
This severely limits the places where parallelism may be introduced, as performing side-effects in parallel can change the meaning of the program.

\subsubsection{Data Parallel Haskell}
Data Parallel Haskell (DPH) is an implementation of nested data parallelism for Haskell\cite{chakravarty2007data}.
The original flattening transform is extended to support higher-order functions\cite{leshchinskiy2006higher}, a common pattern in Haskell,
and arbitrary sum, product and recursive data types\cite{chakravarty2000more}.

Instead of targeting vector machines like \nesl, DPH targets symmetric multi-processor (SMP) machines.
Fusion is performed to remove intermediate arrays, but care must be taken not to reduce parallelism\cite{chakravarty1999portable}.
This is solved by partitioning the program into local operations,
requiring only some small chunk of the array,
and global operations, requiring thread synchronisation\cite{keller1999distributed}.
If only local operations are fused, crossing no global boundaries, no parallelism will be lost.

Unlike \nesl, DPH does not require the entire program to be written using nested data parallelism,
and supports partial vectorisation\cite{chakravarty2008partial}.

The original flattening transform also increases the time and space complexity of the program,
as flattening causes data to be replicated multiple times,
and if fusion does not occur correctly, such intermediate structures will be reified in memory.
The time complexity problem can be fixed by using a clever representation, making replication constant time\cite{lippmeier2012work}.
This does not solve the memory problem, however.

As with all optimisations, conversion from nested data parallelism to flat data parallelism should be semantics preserving:
the original program's meaning must remain unchanged\cite{leshchinskiy2005higher}.
This is complicated by the fact that Haskell is a lazy language, whereas, for efficiency, strict unboxed arrays are used to represent data.
Several tricks are used to preserve laziness while still using strict unboxed arrays.

Flattening converts scalar operations into array operations, even for intermediate scalars\cite{keller2012vectorisation}.
This means that operations that once were simple additions on registers must now work on arrays in memory.
Array fusion can often remove these intermediate structures, but fusion relies heavily on other optimisations such as inlining,
which may not always occur due to sharing. Excessive inlining can also cause code blowup in the compilation process,
leading to long compile times.
Instead of vectorising an entire program and then relying on fusion to remove intermediate arrays,
identifying scalar operations and not vectorising them is less fragile.




%\rev{Scans as Primitive Parallel Operations}{blelloch1989scans}
%If distributed vector machines implemented scan operations on vectors, much time could be saved that would otherwise be spent waiting for memory accesses to return.
%Explanation of radix sort example.
%Segmented scan operations use a vector of flags the same length as the values. I guess this totally disallows empty segments, which makes sense. True flag indicates the start of a segment. What happens if there is no True flag at the start..?
%Clever implementation of Quicksort using segmented scans and iteratively splitting each segment into more subsegments. Using the segment flag representation allows very easy splitting.
%It seems quite disingenuous to claim ``constant time for a vector of size n, provided there are n nodes in the machine''. It does deal with this eventually though, in the load-balancing section.
%Hardware description of a tree scan.

% \rev{A Comparison of Sorting Algorithms for the Connection Machine CM-2}{blelloch1991comparison}
% Different sorting algorithms are implemented and compared for the connection machine. Irrelevant.
% \rev{Nepal - Nested Data-Parallelism in Haskell}{chakravarty2001nepal}
% \rev{Harnessing the Multicores: Nested Data Parallelism in Haskell}{jones2008harnessing}
% \rev{Costing nested array codes}{lechtchinsky2002costing}





\subsection{Feedback directed automatic parallelism}

Automatic parallelism, as in Mercury\cite{bone2010automatic},
is able to parallelise more general programs than nested data parallelism,
but at the cost of compiler simplicity.

Feedback directed parallelism relies on the program being run multiple times,
and each time the program is recompiled.
The compiler uses information from the previous runs to determine the best place to introduce parallelism.

Mercury, a logic language, has many similarities to Haskell: it is pure, statically typed, and has type inference.
The purity allows the compiler to know where it is safe to introduce parallelism.

Automatic parallelism is a far more challenging problem than nested data parallelism,
since the programs do not use any known set of combinators or structure.
For this reason, and because many important scientific problems can be solved with nested data parallelism\cite{blelloch1996programming}, 
automatic parallelism will not be discussed further.


\section{Optimisations}

While the flattening transform for nested data parallelism is quite simple,
it alone cannot achieve performance that competes with that of hand-optimised C code.
This section discusses the optimisations required to get such performance.
Most of these optimisations are general purpose, and applicable to not only nested data parallel programs.

Optimisations for pure languages can make use of the knowledge that arbitrary expressions are side-effect free.
Two classes of optimisations are particularly important for nested data parallel programs in pure languages: fusion and constructor specialisation.
Impure languages must be more conservative in the transformations they apply, in order to preserve the meaning of the program.


\subsection{Fusion}

Fusion merges array producers with consumers, removing the need to allocate intermediate arrays in memory.


\subsubsection{Short-cut fusion}

Short-cut fusion includes techniques such as stream fusion\cite{coutts2007streamfusion},
@fold@/@build@ fusion\cite{gill1993shortcut}
and functional array fusion\cite{chakravarty2001functional, chakravarty2003approach}.
These techniques all work by fusing adjacent producers and consumers together.
This relies on general purpose optimisations such as inlining to bring producers and consumers closer. 

A recent paper shows the possibility of fusion producing SIMD-vector instructions\cite{bik2004software} that operate on multiple elements at a time\cite{mainland2013haskell}.
This allows significant speedups.

There is a balance between inlining and code blowup, however.
Too much inlining and the code will become too large, leading to long compilation times and poor cache performance when a loop kernel does not fit in the instruction line.
As short-cut fusion is a simple local transformation, it can be implemented using general rewrite rule facilities\cite{jones2001playingby} without modifying the compiler.
The local nature of short-cut fusion comes at a price, however, when a single producer has multiple consumers.
As the producer cannot be inlined into both consumers without potentially duplicating work, the producer must be reified in memory.

\subsubsection{Flow fusion}

A new method that we have been working on is known as flow fusion\cite{lippmeier2013flow}.
It uses a more global analysis and can deal with branching dataflows that short-cut fusion cannot.
It analyses an entire function of combinator-based `loops' at a time and schedules the combinators into as few loops as possible.
This analysis is based on Waters' series expressions\cite{waters1991automatic}, using Shivers' loop anatomy\cite{shivers2005anatomy} to merge skeleton code for loops.
This method is similar to compilation for dataflow languages\cite{johnston2004advances}.

This requires more complicated implementation than short-cut fusion, in the form of a compiler plugin.
As it is still very new, no research has been done into SIMD instructions.


\subsubsection{Loop fusion and array contraction}

Fusion in imperative languages is generally expressed as two stages:
loop nests with the same index space are \emph{fused} together,
then intermediate arrays are \emph{contracted} into scalars or smaller buffers.
This requires a \emph{loop dependency graph}\cite{gao1993collective} to be created,
to calculate which loops can be fused without changing the order of writes and reads to arrays.
Loops may have their index spaces reversed or otherwise modified, in order to allow further fusion.
Sarkar shows that the optimal loop configuration can be found with a two-colouring of a graph\cite{sarkar1991optimization}.

Such optimisations may generally increase efficiency, but can in the worst case cause extra cache misses or register spilling.
Loop fusion merges the bodies of two or more loops, which means there are often more local variables
and memory references in the result, all contending for registers and cache.
Some cache misses can be identified by looking at each pair of consecutive requests to the same element, and counting the number of distinct elements requested between the pair.
This is called the \emph{reuse distance}\cite{song2004improving}.
With controlled fusion, loops are only fused together if the maximum reuse distance does not exceed the cache size. 

In cases where a successive loop iteration uses the output of the previous loop iteration, memory reads can be reduced by reusing the register value\cite{wang2013loop}.

Unimodular matrices are used to find new execution orders of loops, where the original index order cannot be parallelised\cite{banerjee1993loop}.
For example, a 2-d loop whose $(x,y)$th iteration depends on the output of its $(x-1,y-1)$th iteration cannot be trivially parallelised.
If the execution order is changed, however, the several threads could work diagonally through the array at the same time.

These imperative optimisations require all loops to only include trivially non-side-effecting statements.
If loops had arbitrary function calls, the compiler cannot generally know whether these function calls had side-effects,
and would not be certain that reordering the calls would be safe.
This problem does not occur in a pure language, as all functions are assured to be side-effect free.


\subsection{Alias analysis}

Another problem that keeps high-performance Haskell from competing with C code is incomplete alias analysis.
After converting variables references to memory reads and writes, superfluous memory traffic can be reduced by storing scalar variables in registers.
This requires that no other variables reference the same memory location, or \emph{alias} with each other,
as converting aliasing variables to registers can change the meaning of the program.
Alias analysis is a low-level optimisation that finds sets of variables known \emph{not} to alias, and removes superfluous memory reads and writes to them.

This can give great improvements in the runtime of loop kernels, as the majority of time would otherwise be spent updating local variables\cite{clifton2012optimisations}.
However, functions generally do not know whether their arguments will alias, so local alias analysis is in some cases doomed to be pessimistic.

Another method is to introduce distinctness witnesses\cite{ma2012type} as proof that two variables will not alias.
These witnesses are introduced and inferred automatically by the compiler, and passed as arguments to functions.
The functions can then be optimised with full assurance that arguments will not alias, leading to faster code.

The Glasgow Haskell Compiler currently only performs rudimentary alias analysis, which leads to problems with tight loops
where the superfluous memory operations outweigh the loop time.

\subsection{Constructor specialisation}

Constructor specialisation removes superfluous allocations from loops\cite{bechet1994limix},
where the allocation is used and thrown away immediately at the start of the next iteration.

In a purely functional language such as Haskell, loops are often expressed as recursive functions, with mutable loop variables becoming function arguments. 
Without mutation, however, any change to the loop variables must be allocated as new objects\cite{peyton2007call}.
This is particularly wasteful when objects are allocated only to be used once in the next iteration, then thrown away.
A solution to this is to specialise recursive functions to remove unnecessary allocation and pattern-matching of constructors.
When a recursive function pattern matches or destructs one of its arguments and makes a recursive call with a constructor of that argument,
the recursive call can be specialised for that particular argument so no allocation or unboxing is necessary.

This has been implemented for both pure and impure languages such as Haskell and ML\cite{thiemann1993avoiding, mogensen1993constructor}.

This kind of compile-time evaluation is also key to modern techniques such as supercompilation\cite{bolingbroke2011supercomp}.
Constructor specialisation is also a generalisation of another technique, worker-wrapper transform\cite{gill2009worker},
that creates a specialised worker function that operates on unboxed data structures.

Excessive specialisation is also a risk, as specialisation increases the code size, potentially leading to longer compilation times and worse cache performance when a loop's code doesn't fit into the cache.
I have been able to reduce this excessive specialisation in the Glasgow Haskell Compiler by first attempting to \emph{seed} the specialisation with the call patterns the function is first called with.
This reduces the number of specialisations to those that will actually be used, instead of generating all possible specialisations.

Other forms of constructor specialisation / supercompilation have been developed that are able to rewrite multiple recursive calls into a single recursion\cite{burstall1977transformation}.
This has impressive results, but it requires user intervention and relies on domain knowledge such as commutativity of addition.




\subsection{Improvement theory}

Improvement theory\cite{sands1998improvement}
classifies program transformations to those that are actually optimisations.
In order to be an optimisation, the result of a transformation must be less than or equally as expensive as the original program.
It must also evaluate to the same result, in that it cannot change the meaning of the program.

In improvement theory, each expression is annotated with a cost, for example the number of function calls required to evaluate the expression.
This cost is not simply syntactic, as one function call may require more function calls, and those must be counted as well.

Lazy, or call-by-need, evaluation works by deferring evaluation of expressions until they are required.
Once expressions have been evaluated once, the value is cached and subsequent evaluations will use this cached value.
This is an improvement when not all expressions are used,
and can also reduce space usage as only part of a value needs to be stored in memory at any point.

Improvement theory has been extended to deal with lazy, or call-by-need, languages\cite{moran1999improvement,gustavsson1999foundation,gustavsson2001possibilities}.
The complication with this is that an expression will not be evaluated more than once.
Thus, the cost of an expression must only take into account the first evaluation of an expression.


\subsection{Flow fusion}

Flow fusion\cite{lippmeier2013flow} is an array fusion method for functional programs
that uses a more global analysis than short-cut fusion.
As a result, it can fuse more interesting dataflow graphs than just the simple straight-line graphs that short-cut fusion can.
It analyses an entire function of combinator-based array operations at a time and schedules the combinators into as few loops as possible.
This analysis is based on Waters' series expressions\cite{waters1991automatic}, using Shivers' loop anatomy\cite{shivers2005anatomy} to merge skeleton code for loops.
This method is similar to compilation for dataflow languages\cite{johnston2004advances}.

As flow fusion is based on Waters' series expressions\cite{waters1991automatic},
has very similar restrictions on the programs that can be fused:
\begin{itemize}
\item preorder: processes inputs and outputs one at a time in ascending order;
\item any cycles must consist only of on-line computations.
\end{itemize}

The preorder restriction is implemented by requiring that all library functions are preorder.
Sadly, this means that many interesting and useful functions
such as @reverse@, @backpermute@ and indexing cannot be implemented,
as they require random access to the array.

On-line computations are those that consume elements from all input streams at the same time,
and then immediately push to their output. 
For example, @map@ and @zipWith@ are on-line as they operate by reading from input, call the mapping function, then produce an output element.
Conversely, @filter@ and @sum@ are off-line;
@filter@ may have to read multiple input elements before producing an output element,
and @sum@ must read the entire input stream before producing its single output.

Off-line computations are not necessarily bad, but mixing on-line and off-line can lead to off-line computations that require unbounded buffers.
In these cases, we fall back to a na\"{i}ve stream fusion system.


It is conceivable that certain cases of non-preorder, off-line functions could be supported,
in the case that one of their arguments was in fact a manifest or otherwise randomly accessible.
For example, @reverse (map f input)@ ought to be supported if @input@ is a manifest array;
it would simply loop through the @input@ array in reverse order, applying @f@ to each element.


\subsection{Space and work complexity issues}

It is a well known problem that flattening, the main idea behind nested data parallelism\cite{blelloch1995nesl}, can cause significantly worse space and work complexity\cite{lippmeier2012work,spoonhower2007semantic}.
The aim of flattening is to transform a nested data parallelism into flat data parallelism,
while maintaining load balancing across processors.

While the work complexity problem has been fixed\cite{lippmeier2012work}, the space problem still persists.
For example, given an array of points, find the maximum distance between all pairs of points.
For each point, we calculate the distance between it and every point, leading to $n$ arrays of $n$ distances, or $n^2$ distances.
We then find the maximum of all these distances.

The unflattened program would allocate $n$ arrays of size $n$ sequentially,
leading to a maximum space usage of only $n$.
However, flattening causes the $n^2$ array to be allocated in memory in its entirety.
Correct fusion would solve this case by fusing the $n^2$ array with @maximum@.
It is unknown, however, whether all such space complexity flaws can be solved by fusion alone.


\pagebreak
\section{Methodology}

\subsection{Instances of the space problem}
The first thing is to classify the different instances of the space complexity problem, with example programs.
As mentioned earlier, flow fusion can fully fuse programs that satisfy certain on-line restrictions\cite{lippmeier2013flow}.
With these programs, we should be able to get a rough idea of how much of the space complexity problem is fixed by flow fusion.
These examples are not strictly necessary, but will help to build an intuition about the problem.

We then want a more rigorous definition of the scope of the problem.
Working backwards from the restrictions of flow fusion,
imagine reversing the flattening. 
The restrictions on the source language are those that, after flattening, only produce flow fusion-able programs.

We then need to find ways of expanding this set, or loosening the restrictions.
For example, off-line combinators like @reverse@ and indexing cannot be fused using flow fusion as they require random access to the input.
While fusion cannot be performed for these in general, certain cases such as reversing over manifest arrays, or simple @map@s or @filter@s of manifest arrays could plausibly be fused optimally.
This intuition must be formalised; perhaps in a similar way to how Repa's type indices\cite{lippmeier2012guiding} separate delayed arrays from manifest arrays.

There is another problem with manifest arrays not being explicit;
sometimes, an unfused program will run faster when aggressive inlining and fusion cause work to be duplicated.
In cases like this, it would be better for the programmer to explicitly indicate which arrays should be manifest and which should be fused away,
rather than obtuse @NOINLINE@ pragmas.


\subsection{Coq theorem prover}
Proofs of correctness are particularly important for optimisations.
An incorrect optimisation may actually make a program more expensive, or change the meaning of the program.
Neither of these errors are acceptable, but sadly compiler transforms can be quite complex and subtle.

For this reason, optimisations should be proved to be correct, for example using improvement theory\cite{sands1998improvement}.
Pen and paper proofs can also lead to problems as it becomes tempting to cut corners,
leading to half-baked or incorrect proofs.

The Coq theorem prover is an interactive theorem prover based on a dependently typed programming language\cite{harper2012practical}.
Interactive theorem provers have the advantage of disallowing such cheating or corner-cutting that proliferates in paper proofs,
as they are machine checked to be valid.
Coq has been used to verify the implementation of an optimising C compiler, CompCert\cite{leroy2012compcert}.
A similar theorem prover, Isabelle, has also been used to formalise higher-order flattening\cite{leshchinskiy2005higher}.

I believe that formal methods such as proofs will only become more necessary as programs continue to become more complicated.
Formalising such transforms and proving them correct will lead to fewer bugs in the implementation.
As these proofs improve the quality of software, I feel it is my responsibility to use these methods wherever applicable.

\subsection{Current progress}

So far, most of my effort has been directed at reviewing the current state of nested data parallelism and high performance functional computing in general.

I have fixed some code blowup problems with the SpecConstr optimisation, and have started writing an article on this for the Monad Reader.
I also intend on giving a talk at the 2013 Haskell Implementor's Workshop in Boston on my SpecConstr work.

I have been working on the implementation of flow fusion, writing the compiler plugin that performs the fusion, and helped write the paper\cite{lippmeier2013flow}.

I have also spent quite a bit of time practicing and learning to write proofs in Coq.
This has also resulted in me learning about the formal definitions of programming languages and their type systems\cite{harper2012practical},
which ties in with a course I did on Concepts of Programming Languages, 3161.

\pagebreak
\section{Research plan}

\paragraph{Semester 1, 2013}
\begin{itemize}
\item Literature review
\item Initial flow fusion work
\item Flow fusion paper accepted to Haskell Symposium\cite{lippmeier2013flow}
\item SpecConstr improvements
\end{itemize}

\paragraph{Semester 2, 2013}
\begin{itemize}
\item Research proposal
\item Attend ICFP and Haskell Symposium
\item Submit talk for Haskell Implementors' Workshop
\item Write article on SpecConstr work for Monad reader
\item Create and classify instances of space complexity problem
\item Low-level optimisations to DPH such as gang scheduling
\end{itemize}

\paragraph{Semester 1, 2014}
\begin{itemize}
\item Formalise which parts of space problem flow fusion does solve
\item Further work on integrating flow fusion with DPH
\item ICFP submission deadline
\end{itemize}

\paragraph{Semester 2, 2014}
\begin{itemize}
\item Formalise flow fusion in Coq
\item Extension of flow fusion to deal with @reverse@ and similar
\item Implement compiler warnings for unfusable programs
\end{itemize}

\paragraph{Semester 1, 2015}
\begin{itemize}
\item Benchmarks
\item Low-level optimisations based on benchmarks
\item Writing
\item ICFP submission deadline
\end{itemize}

\paragraph{Semester 1, 2015}
\begin{itemize}
\item Writing
\item Submission
\end{itemize}


% \pagebreak
% \section{Current progress}

\pagebreak
\input{section/05-Conclusion}

\pagebreak
\section{Draft thesis chapter outline}

\begin{enumerate}
\item Introduction
  \begin{enumerate}
  \item Data parallelism
  \item Fusion
  \item Improvement theory
  \end{enumerate}

\item Data Parallel Haskell
  \begin{enumerate}
  \item Vectorisation
  \item Flattening
  \end{enumerate}

\item Flow fusion
  \begin{enumerate}
  \item Series expressions
  \item Input requirements
  \item Rate inference
  \end{enumerate}

\item Fusing flattened programs
  \begin{enumerate}
  \item The set of perfectly fusible programs
  \item Partial fusion for the rest
  \end{enumerate}

\item Conclusion
\item Appendices
  \begin{enumerate}
  \item Proofs
  \item Source
  \end{enumerate}
\item References
\end{enumerate}

% think the bib can be normal line-spacing
%\singlespacing
\pagebreak

\addcontentsline{toc}{section}{References}
%\bibliographystyle{abbrv}
\bibliographystyle{apalike}
\bibliography{../bibs/Main,../bibs/loops,../bibs/ndp,../bibs/parteval,../bibs/series,../bibs/streams,../bibs/types,../bibs/fusion,../bibs/flat,../bibs/alias,../bibs/improvement,../bibs/spaceflaw}

\end{document}
