\section{Conclusion}

Nested data parallelism is a relatively mature field, but so far it is unable to beat the performance of hand-written C code.
Recent advancements such as flow fusion\cite{lippmeier2013flow} and SIMD support for stream fusion\cite{mainland2013haskell}
continue to converge.
It is unclear, however, whether fusion alone will be able to fix the fundamental memory space problems with flattening\cite{lippmeier2012work}.

Improvements to nested data parallelism such as vectorisation avoidance\cite{keller2012vectorisation}
and work efficient replication\cite{lippmeier2012work} show promise.
Further improvements could be made, however, such as avoiding the replication issue altogether when the same nested workload is shared across all worker threads.

Our new fusion system, flow fusion\cite{lippmeier2013flow}, solves problems with existing fusion such as branching dataflow,
but can only handle specific cases such as \emph{on-line computations} where unbounded buffers are not necessary.
Such cases are common, but by no means exclusive.
For example, consider that reversing an array requires reifying the entire array in memory before being able to retrieve the first element.
Further work is required to gain the best possible fusion in this case, instead of reverting to stream fusion completely.

Other problems with fusion are that excessive fusion can cause register and cache contention\cite{song2004improving}.
While it is true that in these cases, fusion is still better than nothing,
the ideal solution may involve some kind of loop tiling\cite{pike2002better},
where each array is processed in buffers that fit in cache.

Incomplete alias analysis is a low-level issue that only becomes apparent once the rest of the program is perfect.
In other programs, runtime is more often dominated by lack of fusion and memory usage issues.
While runtime would definitely benefit from better alias analysis, my time will be better spent focussing on higher-level issues,
such as better vectorisation avoidance and fusion.


