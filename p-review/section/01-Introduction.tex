
\section{Introduction}
As transistors continue to decrease in size, as they have been for the last fifty years, more transistors are able to fit into a single processor.
Historically this has led to faster processors, allowing programs to take advantage of the increased speed with no effort from the programmer.
Recently, however, this increase in processor speed has come to a halt.
Being unable to increase processor speed, manufacturers have instead started focussing the new transistors on multi-core processors.

The ubiquity of multi-core processors means that sequential programs can no longer take full advantage of the processor.
However, writing parallel programs is significantly more difficult than writing sequential programs.
Nested data parallelism lifts the burden of exploiting multi-cores from the programmer to the compiler-writer.
Current implementations of nested data parallelism use impressive optimisation techniques such as fusion, but are still unable to compete with handwritten parallel code.
The aim of my research is to improve the performance of nested data parallelism, making it easier for programmers to write efficient parallel code.
Because nested data parallelism has focussed on purely functional languages, many well-known imperative optimisations have not been treated.

Purely functional languages, such as Haskell, are more restricted than impure imperative languages
in the sense that arbitrary expressions may not perform side-effects such as destructive updates.
This restriction, however, turns out to have serious benefits:
an optimising compiler is able to reorder expressions to gain better performance,
without any risk of changing the meaning of the program.
From an engineering perspective also, the separation of side-effects from pure computation leads to clearer, easier to understand and verify code.

