\section{Optimisations}

While the flattening transform for nested data parallelism is quite simple,
it alone cannot achieve performance that competes with that of hand-written C code.
This section discusses the optimisations required to get such performance.
Most of these optimisations are general purpose, and applicable to not only nested data parallel programs.

Optimisations for pure languages can make use of the knowledge that arbitrary expressions are side-effect free.
Two classes of optimisations are particularly important for nested data parallel programs in pure languages: fusion and constructor specialisation.
Impure languages must be more conservative in the transformations they apply, in order to preserve the meaning of the program.


\subsection{Fusion}

Fusion merges array producers with consumers, removing the need to allocate intermediate arrays in memory.


\subsubsection{Short-cut fusion}

Short-cut fusion includes techniques such as stream fusion\cite{coutts2007streamfusion},
@fold@/@build@ fusion\cite{gill1993shortcut}
and functional array fusion\cite{chakravarty2001functional, chakravarty2003approach}.
These techniques all work by fusing adjacent producers and consumers together.
This relies on general purpose optimisations such as inlining to bring producers and consumers closer. 

A recent paper shows the possibility of fusion producing SIMD-vector instructions\cite{bik2004software} that operate on multiple elements at a time\cite{mainland2013haskell}.
This allows significant speedups.

There is a balance between inlining and code blowup, however.
Too much inlining and the code will become too large, leading to long compilation times and poor cache performance when a loop kernel does not fit in the instruction line.
As short-cut fusion is a simple local transformation, it can be implemented using general rewrite rule facilities\cite{jones2001playingby} without modifying the compiler.
The local nature of short-cut fusion comes at a price, however, when a single producer has multiple consumers.
As the producer cannot be inlined into both consumers without potentially duplicating work, the producer must be reified in memory.

\subsubsection{Flow fusion}

A new method that we have been working on is known as flow fusion\cite{lippmeier2013flow}.
It uses a more global analysis and can deal with branching dataflows that short-cut fusion cannot.
It analyses an entire function of combinator-based `loops' at a time and schedules the combinators into as few loops as possible.
This analysis is based on Waters' series expressions\cite{waters1991automatic}, using Shivers' loop anatomy\cite{shivers2005anatomy} to merge skeleton code for loops.
This method is similar to compilation for dataflow languages\cite{johnston2004advances}.

This requires more complicated implementation than short-cut fusion, in the form of a compiler plugin.
As it is still very new, no research has been done into SIMD instructions.


\subsubsection{Loop fusion and array contraction}

Fusion in imperative languages is generally expressed as two stages:
loop nests with the same index space are \emph{fused} together,
then intermediate arrays are \emph{contracted} into scalars or smaller buffers.
This requires a \emph{loop dependency graph}\cite{gao1993collective} to be created,
to calculate which loops can be fused without changing the order of writes and reads to arrays.
Loops may have their index spaces reversed or otherwise modified, in order to allow further fusion.
Sarkar shows that the optimal loop configuration can be found with a two-colouring of a graph\cite{sarkar1991optimization}.

Such optimisations may generally increase efficiency, but can in the worst case cause extra cache misses or register spilling.
Loop fusion merges the bodies of two or more loops, which means there are often more local variables
and memory references in the result, all contending for registers and cache.
Some cache misses can be identified by looking at each pair of consecutive requests to the same element, and counting the number of distinct elements requested between the pair.
This is called the \emph{reuse distance}\cite{song2004improving}.
With controlled fusion, loops are only fused together if the maximum reuse distance does not exceed the cache size. 

In cases where a successive loop iteration uses the output of the previous loop iteration, memory reads can be reduced by reusing the register value\cite{wang2013loop}.

Unimodular matrices are used to find new execution orders of loops, where the original index order cannot be parallelised\cite{banerjee1993loop}.
For example, a 2-d loop whose $(x,y)$th iteration depends on the output of its $(x-1,y-1)$th iteration cannot be trivially parallelised.
If the execution order is changed, however, the several threads could work diagonally through the array at the same time.

These imperative optimisations require all loops to only include trivially non-side-effecting statements.
If loops had arbitrary function calls, the compiler cannot generally know whether these function calls had side-effects,
and would not be certain that reordering the calls would be safe.
This problem does not occur in a pure language, as all functions are assured to be side-effect free.

\subsection{Alias analysis}
\TODO{Aliasing problems in flow fusion output}

\TODO{Tran's thesis on distinct witnesses, Robert's thesis?}

\subsection{Constructor specialisation}

Constructor specialisation removes superfluous allocations from loops\cite{bechet1994limix},
where the allocation is used and thrown away immediately at the start of the next iteration.

In a purely functional language such as Haskell, loops are often expressed as recursive functions, with mutable loop variables becoming function arguments. 
Without mutation, however, any change to the loop variables must be allocated as new objects\cite{peyton2007call}.
This is particularly wasteful when objects are allocated only to be used once in the next iteration, then thrown away.
A solution to this is to specialise recursive functions to remove unnecessary allocation and pattern-matching of constructors.
When a recursive function pattern matches or destructs one of its arguments and makes a recursive call with a constructor of that argument,
the recursive call can be specialised for that particular argument so no allocation or unboxing is necessary.

This has been implemented for both pure and impure languages such as Haskell and ML\cite{thiemann1993avoiding, mogensen1993constructor}.

This kind of partial compile-time evaluation is also key to modern techniques such as supercompilation\cite{bolingbroke2011supercomp}.
Constructor specialisation is also a generalisation of another technique, worker-wrapper transform\cite{gill2009worker},
that creates a specialised worker function that operates on unboxed data structures.

Excessive specialisation is also a risk, as specialisation increases the code size, potentially leading to longer compilation times and worse cache performance when the code doesn't fit into the cache line.
I have been able to fix this excessive specialisation in the Glasgow Haskell Compiler by first attempting to \emph{seed} the specialisation with the call patterns the function is first called with.
This reduces the number of specialisations to those that will actually be used, instead of generating all possible specialisations.

Other forms of constructor specialisation / supercompilation have been developed that are able to rewrite multiple recursive calls into a single recursion\cite{burstall1977transformation}.
This has impressive results, but it requires user intervention and relies on domain knowledge such as commutativity of addition.


