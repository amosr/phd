\usepackage{graphicx}
\usepackage{fancyvrb}
\usepackage{amsmath}
\usepackage{amsthm}
\usepackage{fullpage}
\usepackage{float}
\usepackage{alltt}
\usepackage{color}

\floatstyle{plain}
\newfloat{codefig}{H}

\newcommand{\codefigend}{\vspace{-45pt}}

\numberwithin{section}{chapter}
\theoremstyle{plain}    
\newtheorem{thm}{Theorem}[chapter]
\numberwithin{figure}{chapter} %% Comment out for sequentially-numbered
\numberwithin{table}{chapter}  %% Comment out for sequentially-numbered

%-----------------------------------------------------------------------
%  Reduce widow/orphan problems, mainly from a posting from Donald
%  Arsenau on comp.text.tex, 24 Sep 1995.
%  Updated to follow comments from Michael Downes on comp.text.tex,
%  31 Aug 1998.
%-----------------------------------------------------------------------
\doublehyphendemerits=10000     % No consecutive line hyphens.
\brokenpenalty=4991             % Reduce broken words across columns/pages.
\widowpenalty=9999              % Almost no widows at bottom of page.
\clubpenalty=9996               % Almost no orphans at top of page.
\interfootnotelinepenalty=9999  % Almost never break footnotes.
\predisplaypenalty=10000        % Default value
\postdisplaypenalty=1549        % Few breaks between display and widows
\displaywidowpenalty=1602       % At least as high as \postdisplaypenalty

%-----------------------------------------------------------------------
% Change float placement parameters to reduce problems.  Based on
% values posted by Donald Arsenau on comp.text.tex at various times.
% See in particular 17th Nov 1997.
%-----------------------------------------------------------------------
\renewcommand{\topfraction}{.85}
\renewcommand{\bottomfraction}{.7}
\renewcommand{\textfraction}{.15}
\renewcommand{\floatpagefraction}{.66}
\renewcommand{\dbltopfraction}{.66}
\renewcommand{\dblfloatpagefraction}{.66}
\setcounter{topnumber}{9}
\setcounter{bottomnumber}{9}
\setcounter{totalnumber}{20}
\setcounter{dbltopnumber}{9}

%-----------------------------------------------------------------------
% Re-define \cleardoublepage as recommended by Piet van Oostrum in the
% documentation for fancyhdr.sty page 15.  This is to avoid blank pages
% having headers or footers.
%-----------------------------------------------------------------------
\renewcommand{\cleardoublepage}{\clearpage\if@twoside \ifodd\c@page\else
   \thispagestyle{empty}
   \hbox{}\newpage\if@twocolumn\hbox{}\newpage\fi\fi\fi}

%-----------------------------------------------------------------------
% Number figures, tables and equations by chapter.  Re-define footnotes
% and minipage footnotes to be single spaced.  Make new macros needed
% for thesis definitions.
%-----------------------------------------------------------------------
\renewcommand{\thefigure}{\thechapter.\arabic\c@figure}
\renewcommand{\thetable}{\thechapter.\arabic\c@table}
\renewcommand{\theequation}{\thechapter.\arabic\c@equation}

% Re-define \@footnotetext and \@mpfootnotetext to use single spacing
% rather than the space-and-a-half that is the default elsewhere.

\renewcommand{\bibname}{References}

\renewcommand{\title}[1]{\gdef\title{#1}}
\renewcommand{\author}[1]{\gdef\author{#1}}
\newcommand{\school}[1]{\gdef\school{#1}}
\newcommand{\degreetype}[1]{\gdef\degreetype{#1}}
\newcommand{\submitdate}[1]{\gdef\submitdate{#1}}

\renewcommand{\title}{}
\renewcommand{\author}{}
\renewcommand{\school}{Computer Science and Engineering}
\renewcommand{\degreetype}{Bachelor of Computer Science (Hons)}
\newcommand{\university}{The University of New South Wales}
\renewcommand{\submitdate}{\ifcase\the\month\or
  January\or February\or March\or April\or May\or June\or
  July\or August\or September\or October\or November\or December\fi
  \space \number\the\year}
\newcommand{\copyrightyear}{\number\the\year}

%-----------------------------------------------------------------------
% Title page definition.
%-----------------------------------------------------------------------
\newcommand{\titlep}{%
        \thispagestyle{empty}%
        \begin{titlepage}
        \null\vskip2.5cm%
        \begin{center}
                {\rmfamily\Large\expandafter{\title}}
        \end{center}
        \vskip1cm
        \begin{center}
                {\rmfamily\normalsize By \\ \expandafter{\author}}
        \end{center}
        \vfill
        \begin{center}
                \textsc{A thesis submitted for the degree of \\
                \expandafter{\degreetype}}
        \end{center}
        \vfill
        \vskip1cm
\begin{center}
\includegraphics[width=4cm]{template/unswcrest.pdf}
\end{center}
        \vskip1cm
        \begin{center}
                {\rmfamily\normalsize \expandafter{\school},\\
                \expandafter{\university}.}
                \vskip1cm
                {\rmfamily\normalsize \submitdate}\\
        \end{center}
        \vskip1cm
        \end{titlepage}
        \newpage}

%-----------------------------------------------------------------------
% Redefine \maketitle so as to make a UNSW suitable title
%-----------------------------------------------------------------------
\renewcommand{\maketitle}{%
        \thispagestyle{empty}%
        \newpage%
        \titlep
        \raggedbottom
    }

