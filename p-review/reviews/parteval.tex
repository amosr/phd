
\rev{Limix: a partial evaluator for partially static structures}{bechet1994limix}
Splits function specialisation from dead-variable elimination.
This solves a problem with nested structures that ``standard specialisers'' apparently do not treat.
May be some relation to the `reallocation problem' we have in GHC, but I'm not certain about that.

\rev{}{bolingbroke2011supercomp}
\rev{}{bolingbroke2011termination}


\rev{A transformation system for developing recursive programs}{burstall1977transformation}
Very interesting system of unfolding and then folding recursive definitions.
With some cleverness they are able to rewrite several recursive calls into a single recursion.
Seems reminiscent of tupling.
However, it seems to require user intervention and relies on domain knowledge such as commutativity of addition.

\rev{Worker-wrapper}{gill2009worker}
\rev{}{mogensen1993constructor}

\rev{}{peyton2007call}
SpecConstr is an optimisation that specialises recursive functions to remove unnecessary allocation and pattern-matching of constructors.
When a recursive function pattern matches or destructs one of its arguments and makes a recursive call with a constructor of that argument,
the recursive call can be specialised for that particular argument so no allocation or unboxing is necessary.
This is done in three stages;
\begin{enumerate}
\item
Identify the call patterns: recursive function call, with at least one argument being a constructor application that is pattern-matched elsewhere in the body.
\item
Specialise: the function is copied and simplified for the case of that particular constructor argument
\item
Propagate: the existing rewrite rule system is used to find any calls to the functions with known constructors as arguments, and rewritten to call the specialised version.
\end{enumerate}
Additional refinements are used to decide which functions to specialise. This is a trade-off between runtime efficiency and code explosion:
\begin{enumerate}
\item
 take into account variables that have been let-bound or pattern matches, when used as arguments
\item
 record nested pattern matches
\item
 when specialising a function, collect new patterns from the specialised right-hand side
\item
 mutual recursion
\end{enumerate}
There is a potential problem with reboxing: when an argument may be specialised but is used as an argument to another function, specialisation may perform more allocation.




\rev{Avoiding repeated tests in pattern matching}{thiemann1993avoiding}
Very similar to SpecConstr, except for a strict functional language.
Really there's no difference that it's strict.
Explains semantics of a small first-order subset of ML and then shows implementation of specialisation.
First an environment is built up of call patterns of recursive calls, then those cases are specialised.

\rev{}{thiemann1994higher}
\rev{}{thiemann1995generation}

