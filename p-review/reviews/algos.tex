\section{Algorithms}

\rev{A hierarchical $O(n \log n)$ force-calculation algorithm}{barnes1986hierarchical}
This describes a method of efficiently, and relatively accurately, approximating the N-body problem. 
The N-body problem is having $N$ elements all exerting force on each other. A naive approach requires calculating the force between each pair of bodies: $N^2$.

Instead, the problem is distributed into a heirarchical space partitioning tree (quadtree or octree, depending on the dimensions).
Each subtree has an associated centroid describing the centre of gravity of its children.
When calculating the force on a particular body, the tree is explored. Nearby bodies exert force as usual, but more distant subtrees exert force based on their centroid and skip their children.
Calculating the force based on the trees reduces the number of interactions to $N \log N$ instead of $N^2$.
After the force has been calculated for all bodies it is applied, and the tree is recreated using the new positions.

Other hierarchical N-body simulations exist (Appel, Jernigan, Porter) but in these cases the hierarchy does not reflect the spatial relationship of the bodies.
Because of this, the force of nearby bodies may be approximated by centroids. This makes the loss in accuracy hard to analyse.
A better idea of where the accuracy errors are is possible with Barnes-Hut because the octree is an up-to-date space partitioning.

\rev{A modified tree code: don't laugh; it runs}{barnes1990modified}
When this was written in 1988, most powerful supercomputers were vector machines.
To solve N-body for a large dataset, a powerful computer and an efficient method of calculation are required.
This is when tree-based methods really pay off.
However, vector computers are not well suited for recursion over trees because it requires branching.

A modification of Barnes-Hut that is better for vector machines. By sharing some of the `interaction list' between nearby bodies.
Meh.



